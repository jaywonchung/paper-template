\section{Design}\label{sec:design}

Most likely, you want to have some kind of math or theorem in the paper because it looks cool, complicated, and profound.\footnote{I'm not joking. By the way, \textbackslash{}footnote should come \emph{after} the punctuation.}

\begin{equation}
  \label{eq:design-optimization}
  \begin{aligned}
    \textrm{Time}(f)   & = \sum_{i \in C}{t_{i,\textrm{comp}}(f_i)} \\
    \textrm{Energy}(f) & = \sum_{i \in C}{e_{i,\textrm{comp}}(f_i)} + \sum_{i \notin C}{e_{i,\textrm{comp}}(f_i)} \\
  \end{aligned}
\end{equation}
where $f$ is a vector of GPU frequency selections for instructions $\forall i \in \mathcal{G}$ and $C$ is the set of critical operations in $\mathcal{G}$.

\begin{theorem}
  Our problem is NP-hard.
\end{theorem}
\begin{proof}
  Reduction from Knapsack. See Appendix~\ref{sec:appendix-np-hardness}.
\end{proof}

Then, you will also have algorithms.
The paper should always have \emph{both} a text description of what you do and a corresponding algorithm, because some people only read the text whereas others only read the algorithm.

\begin{algorithm}[t]
  \algrule{}

  \KwIn{DAG $\mathcal{G}$ of computations $i \in \mathcal{G}$\newline
        Amount of time to reduce in one iteration $\tau$\newline
        Iteration time with all max frequencies $T_{\min}$
  }
  \KwOut{Set of optimized schedules $\mathcal{S}$}

  \algrule{}

  \Comment{Begin with the minimum energy schedule}
  $s \leftarrow$ Minimum energy for all computations\;\label{algoline:design-min-energy-plan}
  $\mathcal{S} \leftarrow \{ s \}$\;

  \While{$\mathrm{IterationTime}(\mathcal{G}, s) > T_{\min}$}{
    \Comment{Reduce time by $\tau$ with minimal energy increase}
    $s \leftarrow$ GetNextSchedule($\mathcal{G}$, $s$, $\tau$)\label{algoline:design-reducing-time-optimally}\;
    $\mathcal{S} \leftarrow \mathcal{S} \cup \{ s \}$ \;
  }

  \Return{$\mathcal{S}$}

  \algrule{}

  \caption{Iteratively crawling up the frontier.}\label{algo:design-overview}
\end{algorithm}
