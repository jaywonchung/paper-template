\documentclass[letterpaper,twocolumn,10pt]{article}
\usepackage{usenix-2020-09}

\usepackage{times}
\usepackage{amsmath,amsthm}
\usepackage{xspace}
\usepackage{graphicx}
\DeclareMathOperator*{\argmin}{argmin}


% Algorithm style
\definecolor{commentgreen}{rgb}{0, 0.5, 0}
\usepackage[linesnumbered, ]{algorithm2e}
\usepackage{float}
%\usepackage{algorithmicx}
\newcommand\mycommfont[1]{\small\textcolor{commentgreen}{\textrm{#1}}}
\SetCommentSty{mycommfont}
\setlength{\interspacetitleruled}{0pt}%
\setlength{\algotitleheightrule}{0pt}%
\newcommand{\algcomment}[1]{{\footnotesize\Comment{#1}}}
\newcommand{\algrule}[1][.7pt]{\par\vskip.5\baselineskip\hrule height #1\par\vskip.5\baselineskip}


\usepackage[english]{babel}
\usepackage{subfig}
\usepackage{xcolor}
\usepackage{colortbl}
\usepackage{pifont}
\usepackage[inline]{enumitem}
\usepackage{multirow}
\usepackage{fancybox} % for Sbox
\usepackage[small, compact]{titlesec}
\usepackage[textfont={it}, font={small,bf}]{caption}
\usepackage[normalem]{ulem}
\usepackage{authblk}
\usepackage{comment}

% User defined commands
\def\name{Perseus\xspace}

% \newcommand{\revised}[1]{#1}
\newcommand{\revised}[1]{\textcolor{blue}{#1}}
\newcommand{\todo}[1]{\textcolor{red}{\{#1\}}}
% \renewcommand{\mosharaf}[1]{}
\newcommand{\mosharaf}[1]{\textcolor{blue}{\{MC: #1\}}}
% \renewcommand{\jw}[1]{}
\newcommand{\jw}[1]{\textcolor[rgb]{0.25,0.65,0.1}{\{JW: #1\}}}
\newcommand{\topic}[1]{\paragraph{\textit{#1}}}
\newcommand{\suggest}[1]{\textcolor{blue}{\{#1\}}}

\usepackage{tikz}
\newcommand*\circled[1]{\tikz[baseline=(char.base)]{
		\node[shape=circle,draw,inner sep=0pt] (char) {#1};}}
\newcommand*\blackcircled[1]{\tikz[baseline=(char.base)]{
		\node[shape=circle,draw,fill=black,inner sep=0pt] (char) {\textcolor{white}{#1}};}}

% Same thanks, for equal contribution etc.
\newcommand*\samethanks[1][\value{footnote}]{\footnotemark[#1]}

%\newcommand{\argmax}{\operatornamewithlimits{argmax}}

% Math related
\theoremstyle{definition}
\newtheorem{defi}{Definition}
\newtheorem*{defi*}{Definition}

\usepackage{hyperref}
\hypersetup{
	colorlinks=true,      % false: boxed links; true: colored links
	linkcolor=blue,       % color of internal links
	citecolor=magenta,    % color of links to bibliography
	filecolor=cyan,       % color of file links
	urlcolor=red          % color of external links
}

\frenchspacing

\setlength{\textfloatsep}{ 4pt plus 1.0pt minus 2.0pt}
\setlength{\floatsep}    { 4pt plus 1.0pt minus 2.0pt}
\setlength{\intextsep}   { 4pt plus 1.0pt minus 2.0pt}

\newtheorem{lemma}{Lemma}
\usepackage{titling}
\setlength{\droptitle}{-1cm}
\date{}

%\definecolor{boxcolor}{gray}{0.9}
%\newenvironment{colframe}{%
	%	\begin{Sbox}
		%		\begin{minipage}
			%			{0.96\columnwidth}
			%		}{%
			%		\end{minipage}
		%	\end{Sbox}
	%	\begin{center}
		%		\colorbox{boxcolor}{\TheSbox}
		%	\end{center}
	%}

% Code block environment from listing (environment lstlisting)
\usepackage{listings}
\usepackage{courier}
\lstset{
	language=Python,
	basicstyle=\footnotesize\ttfamily,
	commentstyle=\color{commentgreen},
	numbers=left,
	numberstyle=\tiny,
	frame=tb,
	columns=fullflexible,
	frame=bottomline,
	showstringspaces=false,
	captionpos=b,
	keepspaces=true,
}



%\definecolor{codegreen}{rgb}{0,0.6,0}
%\definecolor{codegray}{rgb}{0.5,0.5,0.5}
%\definecolor{codepurple}{rgb}{0.58,0,0.82}
%\definecolor{backcolour}{rgb}{0.95,0.95,0.92}
%\lstdefinestyle{codestyle}{
	%	% backgroundcolor=\color{backcolour},
	%	commentstyle=\color{codegreen},
	%	keywordstyle=\color{black}\bfseries,
	%	numberstyle=\color{codegray},
	%	stringstyle=\color{codepurple},
	%	basicstyle=\ttfamily\mdseries\scriptsize,
	%	emph={selector,kuiper},
	%	emphstyle={\color{magenta}\bfseries},
	%	breakatwhitespace=false,
	%	frame=bottomline,
	%	breaklines=true,
	%	captionpos=b,
	%	keepspaces=true,
	%	numbers=left,
	%	numbersep=5pt,
	%	showspaces=false,
	%	showstringspaces=false,
	%	showtabs=false,
	%	tabsize=2
	%}
%\lstset{style=codestyle}

\newenvironment{denseitemize}{
	\begin{itemize}[topsep=2pt, partopsep=0pt, leftmargin=1.5em]
		\setlength{\itemsep}{2pt}
		\setlength{\parskip}{0pt}
		\setlength{\parsep}{0pt}
	}{\end{itemize}}

\newenvironment{denseenum}{
	\begin{enumerate}[topsep=2pt, partopsep=0pt, leftmargin=1.5em]
		\setlength{\itemsep}{2pt}
		\setlength{\parskip}{0pt}
		\setlength{\parsep}{0pt}
	}{\end{enumerate}}

\def\ie{{i.e.}}
\def\eg{{e.g.}}
\def\etal{{et al.}}
\def\wrt{{w.r.t.}}
\def\etc{etc.}

%-------------------------------------------------------------------------------
\begin{document}
	
	%----- Main Content ------------------------------------------------------------

	%don't want date printed
	\date{}
	
	% make title bold and 14 pt font (Latex default is non-bold, 16 pt)
  \title{\Large \bf Paper Title}

	
	% \author{
	% 	\rm Jie You$^*$ \qquad Jae-Won Chung$^*$ \qquad Mosharaf Chowdhury
	% 	\\
	% 	\itshape{University of Michigan}
	% 	} % end author
	\author{Paper \#XXX, \pageref{EndOfPaper} Pages}
	
	
	\pagestyle{plain}
	
  \pagenumbering{gobble} %% Hides page numbers
	
	\maketitle
	
	A good paper actually \emph{looks} like a good paper.
I mean visually, when you page it through.
Figures are beautiful, equations are typeset nicely, text is well laid out without orphans and widows, and it fills 12 pages perfectly without any extra space.
The majority of pages, if not all, have some kind of floating element like a figure, a table, or an algorithm.


  % In case of equal contribution
  % {\let\thefootnote\relax\footnote{{$^*$Equal contribution.}}}
  % \setcounter{footnote}{0}

	\section{Introduction}\label{sec:intro}

	\section{Motivation}\label{sec:motivation}

	\section{{\name} Overview}\label{sec:overview}

	\section{{\name} Design}\label{sec:design}

	\section{Implementation}\label{sec:impl}

	\section{Evaluation}\label{sec:eval}

	\section{Discussion}\label{sec:discussion}

	\section{Related Work}\label{sec:related}

	\section{Conclusion}\label{sec:conclusion}

	\section*{Acknowledgements}
\label{EndOfPaper}
	
	%----- References --------------------------------------------------------------

	\newpage
	{
		\bibliographystyle{plain}
		\bibliography{ref}
	}
	\clearpage

	%----- Appendices --------------------------------------------------------------

	\appendix

\section{NP-hardness}\label{sec:appendix-np-hardness}

A simplified version of this problem is NP-hard by reduction from the Knapsack problem.

	
\end{document}
%-------------------------------------------------------------------------------
